\newgeometry{left=35mm,right=25mm,top=35mm,bottom=35mm}
\begin{center}
{\Huge \bf Abstract}
\end{center}
\noindent

    Some twenty-five years after the first introduction of joint modelling, clinical trials across multiple disease areas routinely collect information on many longitudinal biomarkers. This represents opportunities and challenges: Multivariate data likely provides better discrimination capabilities from a prediction standpoint, furthermore disregarding the multivariate nature of the data is tantamount to ignoring potentially informative correlations between these longitudinal trajectories; on the other hand, the multidimensional integrals which arise as part of parameter estimation under traditional approaches presents significant computational and statistical difficulties.

    We investigate alternative approaches which enable faster fitting of joint models of survival and \textit{multivariate} longitudinal data. An approximate expectation maximisation algorithm relatively dormant in the literature is repurposed to lessen the computational burden felt by traditional joint models, leading to faster fitting. Furthermore, we extend beyond the typically-used restrictive longitudinal specifications in such models in favour of more flexible, potentially complex, specifications. Extensive simulation studies as well as a thorough application in the disease area of cirrhosis are carried out.

\thispagestyle{empty}
\restoregeometry